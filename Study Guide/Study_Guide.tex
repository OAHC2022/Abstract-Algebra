\documentclass{article}
\usepackage[utf8]{inputenc}
\usepackage[english]{babel}
\usepackage{empheq}
\usepackage{enumitem}
\usepackage{amsthm,amssymb,amsmath}
\usepackage{mathtools} % Bonus
\newtheorem*{remark}{Remark}
\newtheorem*{corollary}{Corollary}
\newtheorem*{theorem}{Theorem}
\newtheorem*{definition}{Def}
\newtheorem*{proposition}{Prop}



\begin{document}

    \section{Binary Operation}
        def: Let A be an aribitrary set, a binary operation is a function: \[f: A \rightarrow A\]
        \begin{itemize}
            \item \textbf{remark:}
            \begin{enumerate}[align=left,leftmargin=15mm]
                \item[1. No exception:] for every orderd pair \((a_1,a_2)\) and \(a_1,a_2 \in A\), there exists a corresponding element
                \begin{itemize}
                    \item Division is not a binary operation on \(\mathbb R\)
                    \item e.g. \((3,0) \in \mathbb R \times \mathbb R\), but \(\frac{3}{0}\) is undefined
                \end{itemize} 
                \item[2. No ambiguity:] for every orderd pair \((a_1,a_2)\) and \(a_1,a_2 \in A\), the corresponding pair will be unique defined.
                \item[3.] Closed under the operation
            \end{enumerate}
            
            \item \textbf{remark:}
                \indent Natural Number starts from 1
        \end{itemize}
    
    \section{Power Set}
        def: let S be an aribitrary set. P(S) is a set consists exactly of all subsets of S (including \(\emptyset\) and S)
        \begin{itemize}
            \item  \textbf{e.g.} if \(S=\{1,2\}\), then \(P(S) = \{\emptyset,\{1\},\{2\},\{1,2\}\}\)\\
            \item  \textbf{Notation:} * , \(\cdot\) , +, or nothing. We also denot *(x,y), \(\cdot\)(x,y), +(x,y), or (x,y)
        \end{itemize}
        

    \section{Addition Properties}
        Axiom: \[+ : A \times A \rightarrow A\]
        \begin{enumerate}
            \item closed under operation
            \item commutative
            \item associative
            \item there exists a neutral element,0, s.t. \(x + 0 = 0 + x = x\)
            \item there is an inverse s.t. \(x + y = y + x = 0\) 
        \end{enumerate}
        \begin{itemize}
            \item \textbf{remark:}
            neutral element is always unique but may not always exists
                \begin{proof}
                    \(e_1 = f(e_1,e_2) = e_2\)
                \end{proof}
            
                \item \textbf{remark:}
            if S has a neutral element, let \(x \in S\). If \(y \in S\) satisfies 
            \(f(x,y)=f(y,x)=e\), then y is called negative of inverse, which is also unique.
            
            \item \textbf{remark:}
            Def associativity, \(\forall x,y,z \in S, f(f(x,y),z) = f(x,f(y,z))\)
                
            \item  \textbf{remark:}
            Def commutativity, \(\forall x,y, f(x,y) = f(y,x)\)
        \end{itemize}
        
    \section{Group}
        def: A set of G if there exists a binary operation * on G such that:
        \begin{enumerate}
            \item there exists a neutral element
            \item there exists a unique inverse for every element in G
            \item associative
        \end{enumerate}
    
    \section{Multiplication Axioms}
    \begin{enumerate}
        \item there exists a neutral element other than 0, called 1
        \item \(\forall x \in \mathbb R, x \neq 0, \exists! y \in \mathbb R\) called the inverse of x
        \item associative
        \item commutative
        \item Distributive Law: 
        Let *, +, be two operations on a set S.
        \begin{itemize}
            \item left distributive: \(x*(y+z) = x*y +x*z\)
            \item right distributive: \((y+z)*x = y*x +z*x\)
        \end{itemize}
    \end{enumerate}

    \begin{itemize}
        \item[\textbf{remark:}]  In \(\mathbb R\), $*$ is distributive to $+$
        \item[\textbf{remark:}]  $R^* = R \setminus \{0\}$ becomes a group under $*$
    \end{itemize}
  
    \section{Order of Group Elements}
    \begin{corollary}
        Let k be an element of a group \(G\). Then \(ord(k)=|<k>|\)\\
        In other words, the order of k is 
        equal to the order of the cyclic group generated by \(k\)
    \end{corollary}
    
    \begin{theorem}
        (Structure of finite cyclic group)\\
        Let G be a cyclic group of \(<x>\) with finite order n.\\
        The following holds:
        \begin{enumerate}
            \item Every subgroup of G is cyclic and is of the shape \(<x^d>\)
            where \(d>0\) and \(d|n\)\\
            More concretely, Let \(d_1,d_2,...,d_r\) be all distinct positive divisor
            of n. then \(<x^{d_1}>,<x^{d_2}>,...,<x^{d_r}>\) exhaust all subgroup of G.
            \item if d and d' are both positive divisors of n, and \(d\neq d'\),
            then \(<x^d>\neq <x^{d'}\)
            \item if \(k \in \mathbb Z\), then \(x^k\) is a generator of G iff \(gcd(k,n)=1\)
            \item \(\forall k \in \mathbb Z\), we have \(<x^k>=<x^d>\), where \(d=gcd(n,k)\)
            \item \(\forall k \in \mathbb Z\), \(ord(x^k)=\frac{n}{gcd(n,k)}\)
        \end{enumerate}
    \end{theorem}
    
    Proof of the theorem above:
    \begin{enumerate}
        \item 
        \begin{proof}Take H be a subgroup of G,.\\
            if \(H=\{e\}\), then \(H=<x^n>\)\\
            if \(H=G\), then \(H=<x>\)\\
            Now let \(H \subseteq G\) be a nontrivial subgroup. 
            Since \(G=\{e,x^1,x^2,...,x^{n-1}\}\), thus H contains some elements 
            of the shape \(x^j\), where \(j\in \{1,2,3,...,n-1\}\).
            Take \(d\in \mathbb N\) be the smallest natural number such that
            \(x^d\in H\).\\
            \textbf{Claim:}\(<x^d>=H\).\\
            1. Since H is a subgroup of G containing \(x^d\). By definition \(<x^d>\)
            is the smallest subgroup of G containing \(x^d\). Thus \[<x^d>\subseteq H\]
            2. Take \(y \in H \Rightarrow y \in G \Rightarrow y = x^m\), where \(m=\{0,1,...,n-1\}\)\\
            By definition using Division Algorithm:
            \[\exists !q,r \in \mathbb Z\] such that \(m=qd+r, 0\leq r < d\)
            \[x^r=x^{m-qd}=x^m(x^{d})^{-q} \Rightarrow x^m=y \in H, (x^d)^{-q}\in H \Rightarrow x^r \in H \]
            If \(r>0\), then r is a natural number strictly smaller than d, also \(x_r\in H\), which contradicts 
            the minimality of d. Thus, \[r=0 \Rightarrow H\subseteq <x^d>\]
            Above all, we know \(H=<x^d>\)\\
            \\Then show that \(d|n\)\\
            Again, use Division Algorithm:
            \[\exists ! q',r'\]
            such that \(n=q'd+r',0\leq r' <d\).
            \[x^{r'}=x^nx{-q'd}=e(x^{d})^{-q'} \in <x^d>\]
            so it contradicts with the minimality of d. Therefore \(r'=0\)
        \end{proof}
        \item
    \end{enumerate}

    \section{Study of Symmetric Group and Permutation Group}
    \begin{definition}
        \textbf{Symmetric Group:} Let A be a non-empty set. Then 
        a symmetric group of A is the group consists of all bijective maps from A to itself
        under the usual compositions of functions.\\
        Notation: sym(A)
    \end{definition}
    
    \begin{definition}
        \textbf{Symmetric Group:} Let A be a non-empty set. Then 
        a permutation group of A is a group G whose elements are 
        bijective maps from A to itself under usual compositions of functions
    \end{definition}

    \begin{remark}
        Symmetric group is unique, because it contains \textbf{all} of the bijective maps
        from A to itself. \\
        Permutation group is not unique, because it contains \textbf{some} of the bijective maps
        from A to itself.\\
        \textbf{A permutation group is a subgroup of symmetric group}
    \end{remark}
    
    \begin{proposition}
        Let \(S_n\) be a permutation group, then \[|S_n|=n!\] 
    \end{proposition}
    \begin{definition}
        \textbf{Cauchy's two line notation:}\\
        e.g. Consider \(\sigma \in S_3\)
        \[\sigma (1) = 2, \sigma (2) = 3, \sigma (3)=1\]
        Notation: \[\sigma = \left( {\begin{array}{ccc}
            1 & 2 & 3\\
            2 & 3 & 1 \\
        \end{array}}  \right)\]
    \end{definition}

    \begin{theorem}
        \textbf{Cayley's theorem}\\
        Every group is isomorphic to permutation group of a certain set.
    \end{theorem}

    \begin{proof}
        Let A be the underlying set of G.\\
        \indent \textbf{Goal:} construct a monomorphism \(T:G \rightarrow sym(A)\)\\
        \indent \(\forall g \in G\), define: \(T_g: A \rightarrow A\)
        such that \(T_g(a)=ga\)\\\\
        \indent \textbf{claim:} \(T_g\) is injective:\\
        \indent Let \(a_1,a_2 \in A\) with \(T_g(a_1)=T_g(a_1)\), 
        so \(ga_1=ga_2 \Rightarrow a_1=a_2\)\\\\
        \indent \textbf{claim:} \(T_g\) is surjective:\\
        \indent Let \(b \in A\), it suffices to find some \(a \in A\),
        such that \(T_g(a)=b\). 
        Thus, \(ga = b \Rightarrow a = g^{-1}b \in A\)\\
        \indent Thus, we define the map \(T:G\rightarrow sym(A)\) 
        to be \(T(g)=T_g\)\\\\
        \indent \textbf{claim:} T is injective:\\
        \indent Let \(g_1,g_2 \in G\)
        \[T(g_1)=T(g_2) \iff T_{g_1}=T_{g_2}\]
        \[T_{g_1}:A\rightarrow A \iff T_{g_2}:A\rightarrow A\]
        \indent as bijective map from A to itself.
        \[\iff T_{g_1}(a)=T_{g_2}(a), \forall a \in A\]
        \indent Take \(a=e_G \Rightarrow T_{g_1}(e_G)=T_{g_2}(e_G)
        \Rightarrow g_1e_G=g_2e_G \Rightarrow g_1 = g_2\)\\\\
        \indent \textbf{claim:} T preserves the group operation:\\
        it suffices to show \(\forall g_1,g_2 \in G, T(g_1,g_2)=T(g_1)
        \circ T(g_2) \iff T_{g_1g_2} = T_{g_1}\circ T_{g_2}\)\\
        \[\forall a \in A, T_{g_1}]\circ T_{g_2}(a) = T_{g_1}(g_2a)
        =g_1g_2a\]
        \[Tg_1g_2(a) = g_1g_2a\]
        Thus, \(T_{g_1g_2} = T_{g_1}\circ T_{g_2}\).\\\\
        Therefore, we have a \(T:G \rightarrow sym(A)\). G is isomorphic 
        to T(G) which is a subgroup of sym(A), which is a permutation group.

    \end{proof}
\end{document}

